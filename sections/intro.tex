	\section{Introduction}
	Deinterleaving temporal data streams is a general machine-learning
	problem with important applications to security and privacy.
	Specifically, interleaved network data streams are a common occurrence
	in cyber-threat monitoring which complicates many analyses.  In many
	instances, the individual stream identifiers are unavailable due
	to technical challenges, such as the vantage point of the data
	collector or are intentionally supressed to protect the privacy
	of the users in the network.
	
	For example, consider packet traces collected in a local area network,
	where the source IP addresses are removed or data collected from the
	external-facing interface of a proxy server or NAT firewall
	where individual client identifiers are unavailable.  Detecting anomalous
	behavior, especially stealthy and low-volume attack patterns, in
	these aggregated noisy streams is significantly more challenging than
	in a traditional deinterleaved setting.
	
	In this paper, we discuss a variant of this problem, i.e.,
	deinterleaving client request streams from recursive DNS resolvers
	to mine threat intelligence.  Such DNS data streams are shared
	among Internet service providers (ISPs) through mediums such
	as the Security Information Exchange (SIE) and are a valuable
	source of intelligence to the cybersecurity community.  Here, the individual
	client requests to the recursive DNS resolver are typically suppressed and
	what we have are inter-resolver communications (i.e, communications
	between the recursive resolver and the root server, TLD servers
	and other secondary resolvers).  We are interested in
	the application of advanced machine-learning techniques to automate
	the extraction of malware domain groups~\cite{} from such resolver streams.
	
	Malware infections while browsing the Internet have become very
	prevalent and occur due to various reasons such as drive-by exploits,
	phishing attacks etc.  In a typical infection, the user starts from
	a landing page and then goes through a sequence of seemingly harmless
	intermediate websites, until reaching a site that contains the malicious
	exploit that harm the user by installing malware or stealing private
	data.  The intermediate sites are typically redirection chains implemented
	in JavaScript for the purpose of obfuscation.
	Even though many landing and exploit websites are continously identified and
	blacklisted, thousands of new malicious domains emerge daily.  However,
	pieces of the redirection infrastructure get reused across campaigns
	and thus the actual sequence of websites traversed by the user contains
	information that may help in quickly identifying new exploit sites.
	
	When a user makes a browser request to visit a website, it first
	resolves the domain name by asking its recursive resolver.  If the
	answer for the query is cached by the resolver the answer is
	immediately provided to the client. Otherwise, it initiates a set of
	recursive queries, leading to the final queried website's IP address.
	Each webpage may have several embedded objects from many domains
	leading to a sequence of domain lookup requests emanating from the
	client.  Tracking the set of DNS requests made by each client is thus
	a useful means to identifying new and emergent malware infection
	sequences.  However, to protect user privacy ISPs typically only
	capture data from the external facing interface of the recursive
	resolver, effectively suppressing the individual client stream
	identifiers.  As there are hundreds of users making requests at the
	same period of time, and all of these requests are pushed to a single
	queue of a local DNS resolver, we cannot tell apart individual user's
	sequences of requests and perfectly deinterleaving all requests for
	deanonymization purposes is impossible.  However, our objective is not
	deanonymization, but rather extraction of malware domain sequences
	which are observed repeatedly across resolvers.  We believe that advanced
	machine learning strategies could be  in such selective
	deinterleaving of DNS time-series for the extraction of malware
	domain groups.
	
	
	{\bf Prior Work. } To the best of our knowledge deinterleaving has not
	been applied to DNS resolver queue's data.  Some earlier
	work \cite{gao2013empirical} investigates the use of a sliding window
	approach to identify new malicious domains by exploring the domains
	that typically form neighbors of known malicious domains in the
	resolver queue, while ignoring the actual sequential information.  The
	challenges of applying existing deinterleaving methods to DNS data is
	twofold.  First, most of the methods has been designed for
	deinterleaving Markov chains \cite{batu2004inferring,
		seroussi2009deinterleaving, seroussi2012deinterleaving,
		minot2014separation} and HMMs \cite{landwehr2008modeling}, and as we
	will discuss in Section ?, the dynamic of submitting new queries to
	the local resolver is more complicated than simple Markov chain or
	HMM.  Moreover, the state space of the models and number of sequence
	sources are very small in previous work
	applications \cite{minot2014separation, landwehr2008modeling}, while
	in our application, huge number of websites explodes the size of state
	space and also tens of users may be active in a network
	simultaneously.  Because of the nature of our dataset, we need to use
	tools other than those adopted in
	literature \cite{minot2014separation,
		landwehr2008modeling,burge1998finding, burge1997prediction}.
	
	Another very useful model for time-series is Recurrent Neural Networks
	(RNN).  Recently, RNNs and their variants (Gated Recurrent Units
	(GRUs)\cite{chung2014empirical}, Long Short-Term Memory
	(LSTMs) \cite{Hochreiter}) have seen a lot of success in modeling
	time-series in multiple domains \cite{bahdanau2014neural,
		NIPS2008_3449, sutskever2014sequence}.  However, to our knowledge even
	simple RNN tools have not been applied to the deinterleaving
	problem. Using RNN-type tools for deinterleaving mixed DNS request
	logs is a completely unexplored area.  Motivated by the power of RNNs
	to model non-linear dependencies, we seek to apply RNNs to such data
	and start a new direction of work towards identifying newer malicious
	domains more efficiently.
	
	% setting of DNS request logs as time-series.
	{\bf Contributions. } This paper presents a preliminary exploration of
	the utility of various machine-learning models to address the time
	series deinterleaving problem for malware domain group extraction.
	Specifically, we present a model for DNS request generation and
	resolver-sequence interleaving and evaluate the utility of various
	inference strategies on sythetic examples including Viterbi, RNNs and
	LSTMs finding that LSTMs outperform simple RNNs and Viterbi.  Extending
	this analysis to real and large-scale datasets is future work. 
